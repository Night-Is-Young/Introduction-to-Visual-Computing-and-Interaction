\documentclass{ctexart}
\usepackage{Note}
\begin{document}
\section{物理模拟}
万事万物的运动都遵循一定的原理.经过物理学家们长久以来的努力,我们已经可以用一系列物理方程描述物体的运动,但这些方程往往复杂而难以求解.于是,人们开始寻求这些方程的近似解,并且希望这些近似解足够精确,以至于可以用来模拟现实世界中的物理现象.这种通过数值方法求解物理方程,并用计算机进行模拟的过程,就称为\tbf{物理模拟(Physical Simulation)}.
\subsection{弹簧质点模型}
在本节中,我们将以弹簧质点模型为例介绍物理模拟的一般方法.同时,这一模型也是计算机图形学中常用的物理模拟模型之一,研究其仿真办法也具有相当的实际意义.
\subsubsection{物理模型}
一般而言,物理模拟的范围主要是宏观低速物体的运动,因此主要使用牛顿运动定律进行运动的计算.牛顿运动定律为
\[\vec{f}=m\vec{a}=m\dfrac{\di\vec{v}}{\di t}=m\dfrac{\di^2\vec{x}}{\di t^2}\]
\subsubsection{物理模拟}
在上一小节中,我们的研究对象是一个质点,并且它仅受一个力.这个例子实在过于简单,无法代表一般的情形.通常而言,物体包含无穷多粒子,并且往往存在多个外力以及复杂的内力.为了将这样的复杂的系统变为数值求解问题,我们首先需要对物体进行离散化处理.\\
\indent 因此,物理模拟的过程大致可以分为三个步骤:\tbf{空间离散化}, \tbf{时间离散化}以及\tbf{数值计算}.
\subsubsection{空间离散化}
通过一定的方法,我们可以把物体抽象成由有限个质点组成的系统.单个质点的状态由其位置$\vec{x}_i$和速度$\vec{v}_i$,而为了考虑其运动状态的变化,我们还需通过其受的力$\vec{f}_i$和质量$m_i$得出其加速度.因此,一个质点模型可以由列表$\left\{\vec{x}_i,\vec{v}_i,\vec{f}_i,m_i\right\}_{i=1}^{N}$表示.\\
\indent 在抽象成质点的情形下,弹簧质点模型假定质点之间存在弹簧以传递内力.质点$i$受的外力可以写做
\[\vec{f}_i=\sum_{j\in\Omega(i)}\vec{f}_{ij}+f_{i}^{\text{ext}}\]
其中$f_{i}^{\text{ext}}$表示质点$i$受到的除去弹簧弹力之外的外力.质点$i$与其邻接的质点$j\in\Omega(i)$通过弹簧相连,弹簧对质点$i$的弹力$\vec{f}_{ij}$可以通过胡克定律计算:
\[\vec{f}_{ij}=-k_{ij}(\lvert\vec{x}_i-\vec{x}_j\rvert-l_{ij})\dfrac{\vec{x}_i-\vec{x}_j}{\lvert\vec{x}_i-\vec{x}_j\rvert}\]
其中$k_{ij}$为弹簧的弹性系数, $l_{ij}$为弹簧的原长.
\subsubsection{时间离散化}
物体的运动状态是随时间连续变化的.同样地,我们也需要对时间进行离散化.一般而言,我们会在时间区间上均匀采样,采样间隔为$\Delta t$.我们把$N$个质点的位置$\vec{x}$和速度$\vec{v}$写做$3N$维的堆叠向量的形式,在第$k$次采样$t_k$时有:
\[\mat{x}(t_k)=\begin{bmatrix}
    \vec{x}_{1}(t_k)\\\cdots\\\vec{x}_{n}(t_k)
\end{bmatrix},\quad\mat{v}(t_k)=\begin{bmatrix}
    \vec{v}_{1}(t_k)\\\cdots\\\vec{v}_{n}(t_k)
\end{bmatrix}\]
其中$\vec{x}_i(t)$表示$t$时刻质点$i$的位置, $\vec{v}_i(t)$同理.根据简单的运动学知识,我们可以得到下一次采样时的位置矩阵$\vec{x}(t_{k+1})$和$\vec{v}(t_{k+1})$:
\[\vec{x}(t_{k+1})=\vec{x}(t_k)+\int_{t_k}^{t_{k+1}}\vec{v}(t)\di t\]
\[\vec{v}(t_{k+1})=\vec{v}(t_k)+\mat{M}^{-1}\int_{t_k}^{t_{k+1}}\vec{f}(t,\vec{x}(t),\vec{v}(t))\di t\]
其中$\mat{M}=\diag\{m_1,m_1,m_1,\cdots,m_N,m_N,m_N\}$为质点的质量矩阵.\\
\indent 现在,我们的目标就是求解上述积分方程.典型的计算时间积分的办法有\tbf{显式欧拉(Explicit Euler)}积分和\tbf{隐式欧拉(Implicit Euler)}积分.
\paragraph{显式欧拉积分}
在复杂的情形中,上述积分方程可能不存在解析表达.好在当取样的时间间隔较小时,我们可以将被积函数视作常数,取值即为\textit{每个时间间隔开始的时刻},于是即有
\[\vec{x}(t_{k+1})=\vec{x}(t_k)+\vec{v}(t_k)\Delta t\]
\[\vec{v}(t_{k+1})=\vec{v}(t_k)+\mat{M}^{-1}\vec{f}(t_{k})\Delta t\]
这样,对于弹簧质点系统的模拟就归结于上述迭代过程.显式欧拉积分的优点在于计算和逻辑简单,每一步只需要计算当前时刻的力即可.\\
\indent 然而,这样的简单近似也有其缺点.当时间步长较大时, $\vec{v}(t)$和$\vec{f}(t)$可能在时间间隔内变化较大,从而导致解得的$\vec{x}(t)$和$\vec{v}(t)$偏离实际情况,使得系统能量升高.这又进一步导致$\vec{v}(t)$和$\vec{f}(t)$的偏离,最终可能导致系统发散,无法继续模拟下去.\\
\indent 解决上述问题的一种办法是使用更小的时间步长,但这会大大增加计算量,反而得不偿失.因此,我们需要一种更稳定的办法.
\paragraph{隐式欧拉积分}
与显式欧拉积分不同,隐式欧拉积分在计算下一个时间步长的状态时,使用的是\textit{每个时间间隔结束的时刻}的力.即有
\[\vec{x}(t_{k+1})=\vec{x}(t_k)+\vec{v}(t_{k+1})\Delta t\]
\[\vec{v}(t_{k+1})=\vec{v}(t_k)+\mat{M}^{-1}\vec{f}(t_{k+1})\Delta t\]
这样,我们需要解一个隐式方程组.将$\vec{v}(t_{k+1})$代入$\vec{x}(t_{k+1})$中可得
\[\vec{x}(t_{k+1})=\vec{x}(t_k)+\left(\vec{v}(t_k)+\mat{M}^{-1}\vec{f}(t_{k+1})\Delta t\right)\Delta t\]
为了方便考虑,我们把力$\vec{f}$分成内力和外力两部分:
\[\vec{f}(t_{k+1})=\vec{f}_{\text{int}}(t_{k+1})+\vec{f}_{\text{ext}}\]
其中$\vec{f}_{\text{ext}}$作为外力与时间和质点的位置无关.于是进一步整理可得
\[\vec{x}(t_{k+1})=\left[\vec{x}(t_k)+\Delta t\vec{v}(t_k)+(\Delta t)^2\mat{M}^{-1}\vec{f}_{\text{ext}}\right]+(\Delta t)^2\mat{M}^{-1}\vec{f}_{\text{int}}(t_{k+1})\]
等号右边前半部分全部为已知量,不妨记作$\vec{y}(t_k)$.于是方程组即为
\[\vec{x}(t_{k+1})-\vec{y}(t_k)-(\Delta t)^2\mat{M}^{-1}\vec{f}_{\text{int}}(t_{k+1})=\mbf{0}\]
接下来,我们需要将内力$\vec{f}_{\text{int}}(t_{k+1})$表示成位置$\vec{x}(t_{k+1})$的函数.对于质点$i$与$j$之间的弹簧,其弹性势能
\[E_{ij}=\dfrac12k_{ij}\left(\left|\left|\vec{x}_j-\vec{x}_i\right|\right|-l_{ij}\right)^2\]
于是
\[\vec{f}_{ij}=-\dfrac{\p E_{ij}}{\p\vec{x}_i}\]
从而对于质点$i$,其内力$\vec{f}_{i,\text{int}}$就可以写作
\[\vec{f}_{i,\text{int}}=-\sum_{j\in\Omega(i)}\dfrac{\p E_{ij}}{\p\vec{x}_i}=-\dfrac{\p E}{\p\vec{x}_i}\]
其中$E$为弹簧系统的总能量(与$i$不相连的弹簧的势能对$\vec{x}_i$的梯度为$\mbf{0}$,因此可以并入求和项中),其自变量为各质点的位置向量$\vec{x}$.将所有质点的内力堆叠起来,我们就有
\[\vec{x}(t_{k+1})-\vec{y}(t_k)+(\Delta t)^2\mat{M}^{-1}\dfrac{\p E(\vec{x}(t_{k+1}))}{\p \vec{x}}=\mbf{0}\]
于是,可以构造辅助函数
\[g(\vec{x})=\dfrac{1}{2(\Delta t)^2}(\vec{x}-\vec{y}(t_k))^\t\mat{M}(\vec{x}-\vec{y}(t_k))+E(\vec{x})\]
则有
\[\dfrac{\p g(\vec{x})}{\p\vec{x}}=\vec{x}(t_{k+1})-\vec{y}(t_k)+(\Delta t)^2\mat{M}^{-1}\dfrac{\p E(\vec{x}(t_{k+1}))}{\p \vec{x}}=\mbf{0}\]
于是,目标就转化为最小化问题
\[\vec{x}(t_{k+1})=\arg\min_{\vec{x}}g(\vec{x})\]
\indent 对于上述最小化问题,使用梯度下降法可能不是一个很好的选择,因为$g(\vec{x})$很可能是非凸的.我们采取牛顿法解决上述问题.\\
\indent 牛顿法的基本思想是在每轮迭代中使用一个二次函数逼近目标函数,然后求解该二次函数的极小值作为下一次迭代的点.迭代过程每次通过尝试解$\vec{x}^{(k)}$计算下一次
\subsection{流体模拟}
\subsubsection{流体的基本性质}
\begin{theorem}[Navier-Stokes方程]
    流体的运动遵循Navier-Stokes方程,其表达式为
    \[\rho\dfrac{\text{D}\vec{v}}{\text{D}t}=-\nabla p+\rho\vec{g}+\mu\nabla^2\vec{v}\]
    其中$\dfrac{\text{D}v}{\text{D}t}$为流体的随体导数,其定义为
    \[\dfrac{\text{D}\vec{v}}{\text{D}t}=\dfrac{\p\vec{v}}{\p t}+(\vec{v}\cdot\nabla)\vec{v}\]
\end{theorem}
\begin{theorem}[不可压性质]
    流体中的质量守恒方程为
    \[\dfrac{\p\rho}{\p t}=-\nabla\cdot(\rho\vec{v})\]
    一般的流体模拟中总是假设流体不可压缩/膨胀,于是密度$\rho$在各处和任意时刻均为常数.于是上式即为
    \[\nabla\cdot\vec{v}=0\]
    即流体的速度场是无散的,也即流入任意一点的流体体积总是等于流出该点的流体体积.
\end{theorem}
\subsubsection{描述流体的两种方式}
上述N-S方程是以\tbf{欧拉视角}描述流体的运动的.欧拉视角是指将流体的所有物理量看成空间上的一个场,然后描述这个场随时间的变化.这好比在水中插有无限多的木桩,每个木桩上装有检测水的流速,密度,压强等的传感器,我们描述的是所有传感器上的数值变化.\\
\indent 另一种描述流体的方式是\tbf{拉格朗日视角},即将流体看成由无数个质点组成,然后描述每个质点随时间的运动.这好比在水中放入无数悬浮的小球,其上也装有传感器,我们描述的是每个小球的位置,速度等随时间的变化.
\subsubsection{光滑粒子流体}
我们首先采用拉格朗日视角来描述流体的运动.
\begin{theorem}[光滑粒子流体模型]
    光滑粒子流体(Smoothed Particle Hydrodynamics, SPH)模型将流体看成有限个流体微团(视作粒子)组成的系统.每个粒子$i$具有位置$\vec{x}_i$,速度$\vec{v}_i$,质量$m_i$.
\end{theorem}
为了根据上述信息计算流体的宏观性质,例如求解N-S方程必不可少的密度场和压强场,我们采用一种经典的办法:\tbf{核函数}.
\paragraph{核函数}
我们已经经历了很多采样问题以从离散的样本点重建连续函数.对于流体而言,如果采取简单的近邻法划分空间并计算目标函数,往往会在边界上造成函数值的跳变,因而结果交叉.因此,我们需要采取一定的办法使得这种不连续性被平滑.一种自然的想法就是考虑目标点附近的所有粒子对该点性质的影响,并且距离越近的粒子影响越大.这就可以通过引入核函数$W(\vec{r},h)$进行加权平均而解决.
\end{document}