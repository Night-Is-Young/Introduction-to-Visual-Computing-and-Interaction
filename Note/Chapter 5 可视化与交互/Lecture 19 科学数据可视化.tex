\documentclass{ctexart}
\usepackage{Note}
\begin{document}
\section{科学数据可视化}
\subsection{场的可视化}
在科学计算和工程分析中,物理量通常以不同的数学形式存在,并且在空间中构成特定的场.如何将这些场进行可视化就是本节的主要内容.
\subsubsection{场的分类}
\begin{definition}[标量场]
    \tbf{标量场(Scalar Field)}是指在空间中每一点都对应一个标量值的场.温度场,压力场等都是典型的标量场.
\end{definition}
\begin{definition}[向量场]
    \tbf{向量场(Vector Field)}是指在空间中每一点都对应一个向量值的场.速度场,力场等都是典型的向量场.
\end{definition}
\begin{definition}[张量场]
    \tbf{张量场(Tensor Field)}是指在空间中每一点都对应一个张量值的场.应力场,应变场等都是典型的张量场.
\end{definition}
\subsubsection{标量场的可视化方法}
相对而言,标量场的可视化方法比较简单.
\paragraph{颜色映射}
\tbf{颜色映射(Color Mapping)}是将标量值映射到颜色空间的一种方法.这在本质上是各种插值方法的应用,因此不做过多赘述.
\paragraph{等值面与等值线}
等值面和等值线的提取可以用我们在\tbf{几何重建}中讲到的行进立方体算法.这里也不做过多赘述.
\subsubsection{向量场的可视化方法}
\paragraph{箭头表示法}
对于向量场的可视化,使用\tbf{箭头(arrows)}表示某处的向量是最简单直接的方式.对于定义在域$D$上的矢量场$\vec{v}:D\to\R^n$表示每个$\vec{x}\in D$的矢量值,我们在$D$上进行采样(通常是网格点),并在每个采样点处绘制一个箭头.箭头的起点为采样点,方向和长度分别与矢量的方向和大小成正比.最终,向量场$\vec{v}$的可视化$\hat{\vec{v}}$是在采样点上的一组箭头.
\paragraph{流场可视化}
在向量场可视化中,流场(例如水流,气流等)可视化是常见且重要的一种应用.我们先来介绍流场可视化中的基本物理量.
\begin{definition}[流线]
    \tbf{流线(Streamline)}是指在流场中,与流体速度方向切线方向一致的曲线.流线可以表示流体在某一时刻的瞬时运动轨迹.
\end{definition}
\begin{definition}[迹线]
    \tbf{迹线(Traceline)}是指同一空间位置持续释放的流体粒子在流场中随时间运动所形成的轨迹.迹线反映了流体粒子的历史运动路径.
\end{definition}
\begin{definition}[路径线]
    \tbf{路径线(Pathline)}是指在流场中,某一特定流体粒子随时间运动所形成的轨迹.路径线描述了单个流体粒子的运动过程.
\end{definition}
\subsection{表面数据和体积数据的可视化}
\subsubsection{表面数据的可视化}
容易想到表面数据的可视化可以按照渲染的方法进行,而我们已经介绍了一整套成熟的渲染方法.此外,表面数据主要依托于纹理进行存储,因此这也依赖于纹理映射技术.总之,这部分内容和渲染是高度重合的,这里也不做过多赘述.
\subsubsection{体积数据的可视化}
对于体积数据的可视化,当然可以将它们转换为表面数据并按照正常的过程渲染,但我们希望通过某种手段直接渲染体数据,使得视线能穿透表面直达内部.这种方法被称作\tbf{体渲染(Volume Rendering)}.\\
\indent 我们现在介绍体渲染的几种常用的方法.
\paragraph{光线投射}
和我们在光线追踪中介绍的类似,光线投射应用于体渲染时,也是从视点出发,沿着视线方向采样体数据,并将采样结果进行累积以得到最终的像素值.具体步骤如下:
\begin{enumerate}[label=\tbf{\arabic*.}]
    \item \tbf{光线生成}:对于图像中的每一个像素\footnote{如果需要进行超采样则是对每一个采样点},从视点出发沿着该像素对应的方向生成一条光线.
    \item \tbf{光线求交}:沿着光线前进的方向按照一定的步长\footnote{可以采用固定的均匀波长,也可以采用自适应的步长.}对体数据进行采样,得到一系列采样点.如果采样点不在整数坐标处\footnote{这里假定体数据按照离散的立方体网格的形式存储在每个整数格点上,因此非整数处需要插值计算.},则需要进行插值计算以得到该点的值.
\end{enumerate}
\end{document}