\documentclass{ctexart}
\usepackage{Note}
\begin{document}
\section{图形管线}
我们已经大致学习了从二维图形的绘制,再到三维图形的建模,再到本章的三维场景的渲染.现代的大型实时游戏场景中包含大量的三角形面片,并且为了看起来流畅,每秒渲染的帧数常常需要几十上百帧.为了高效完成图形渲染的工作,人们对渲染的流程进行了长时间的优化和演进,最终形成了\tbf{图形管线(Graphic Pipeline)}的概念.
\begin{definition}[图形管线]
    \tbf{图形管线}是在GPU/图形硬件层面上,将顶点与图元的输入经过一系列确定的硬件/着色器阶段(顶点处理,图元装配,裁剪/投影,光栅化,片段处理,像素测试与输出合并等),以产生最终帧缓冲中像素值的有序数据处理流水线.
\end{definition}
通常,我们所说的\tbf{渲染管线(Rendering Pipeline)}大致与图形管线相同.
\subsection{渲染管线的一般过程}
通常,渲染管线可以分为以下几个步骤,每一个步骤接受的输入都是上一步骤的输出.
\begin{enumerate}
    \item \tbf{应用(Application)}
    \begin{itemize}
        \item \textit{目的}:构建和提交需要渲染的几何数据和渲染状态.
        \item \textit{输出}:由需要渲染的3D应用(例如游戏)给出,包括模型数据,纹理数据,相机参数,光源信息等等\footnote{通常,在渲染过程中采取的加速算法也由CPU执行,因此包含在此部分内.}.
    \end{itemize}
    \item \tbf{顶点处理(Vertex Processing)}
    \begin{itemize}
        \item \textit{目的}:将模型变换到裁剪空间\footnote{通过我们在几何变换中所学的投影变换的方法转换到裁剪空间中.裁剪空间中顶点以齐次坐标的形式表示,并且第四个分量$w$尚未进行归一化.对裁剪空间中顶点坐标做下述变换
        \[\begin{bmatrix}
            x&y&z&w
        \end{bmatrix}^{\text{t}}\longrightarrow\begin{bmatrix}
            \dfrac xw&\dfrac yw&\dfrac zw
        \end{bmatrix}^{\text{t}}\]
        即归一化并除去第四个分量后才能转换成规范化坐标系, 即\tbf{NDC(Normalized Device Coordinates)空间}.这一步被称作\tbf{透视除法},是下一步三角形处理中的操作.};然后计算顶点处的各种属性(包括法线, UV坐标,颜色等).
        \item \textit{输出}:顶点在裁剪空间中的位置及其它属性.
    \end{itemize}
    \item \tbf{三角形处理(Triangle Processing)}
    \begin{itemize}
        \item \textit{目的}:将顶点组装成三角形图元,并进行裁剪和投影变换.
        \item \textit{过程}:\begin{enumerate}[label=\tbf{\alph*.}]
            \item \tbf{图元装配}:将顶点按预定的模式组装成三角形图元.
            \item \tbf{背面剔除}:基于顶点顺序与法线方向的关系删除不可见的三角形.
            \item \tbf{裁剪}:将超出视锥的三角形裁剪裁剪成视锥内的部分\footnote{这一步就是裁剪空间这一名称的由来.}.
            \item \tbf{透视除法与视口变换}:将裁剪空间中的顶点通过透视除法转换到NDC空间,然后通过视口变换映射到屏幕空间.
        \end{enumerate}
        \item \textit{输出}:屏幕空间中的三角形片元(包含各种属性).
    \end{itemize}
    \item \tbf{光栅化(Rasterization)}
    \begin{itemize}
        \item \textit{目的}:将三角形片元转换为屏幕的片元(Fragments, 可以理解为采样点),并生成用于插值的片元数据\footnote{一些超采样算法,例如MSAA,需要在此时进行覆盖测试;此外,片元的插值需要经过透视矫正.}.
        \item \textit{输出}:片元流(Fragment Stream),其中每个片元包含插值后的属性(颜色,法线, UV坐标,深度等).
    \end{itemize}
    \item \tbf{片元处理(Fragment Processing)}
    \begin{itemize}
        \item \textit{目的}:通过给定的着色器,以及片元的各种数据(包括纹理坐标,法线,顶点颜色,深度等)计算片元的最终颜色值.
        \item \textit{输出}:每个片元的最终颜色值和深度值.
    \end{itemize}
    \item \tbf{帧缓冲操作(Framebuffer Operations)}
    \begin{itemize}
        \item \textit{目的}:将片元的颜色和深度值与帧缓冲中的现有值进行测试和合并,以生成最终的像素值.
        \item \textit{过程}:
        \begin{enumerate}[label=\tbf{\alph*.}]
            \item \tbf{深度测试}:比较片元的深度值与帧缓冲中对应像素的深度值,以确定片元是否可见.
            \item \tbf{模板测试}:基于模板缓冲的值决定片元是否应被绘制.通常用于复杂遮罩,轮廓等高级效果的控制.
            \item \tbf{混合}:将片元的颜色值与帧缓冲中现有的颜色值进行混合(如果需要),以实现透明度等效果.
        \end{enumerate}
    \end{itemize}
    \item \tbf{显示(Display)}
    \begin{itemize}
        \item \textit{目的}:将帧缓冲中的最终像素值传输到显示设备(例如监视器)以进行可视化.
        \item \textit{输出}:显示设备上呈现的最终图像.
    \end{itemize}
\end{enumerate}
\subsection{图形API}
为了实现对图形管线的控制和使用,现代计算机图形学中引入了\tbf{图形API(Graphic API)}的概念.
\begin{definition}[图形API]
    \tbf{图形API}是指应用程序与图形硬件之间的抽象接口,它通过一组函数或指令集,使开发者能够控制GPU执行图形渲染,计算和资源管理等操作,而无需直接编写底层驱动代码.\\
    常见图形API包括OpenGL, Vulkan, DirectX, Metal等.
\end{definition}
上面介绍的渲染管线仅为通用的大致结构,各个图形API对管线的具体实现过程是不同的.以OpenGL为例,其渲染管线如下所示.
\begin{figure}[H]\centering
\begin{tikzpicture}[
    node distance=0.5cm,
    box/.style={rectangle, minimum width=6cm, minimum height=1cm, text centered, draw=black, top color=yellow!30, bottom color=yellow!70},
    optbox/.style={rectangle, minimum width=6cm, minimum height=1cm, text centered, draw=black, dashed, top color=blue!20, bottom color=blue!40},
    arrow/.style={-latex, thick}
]
    \node (vspec) [box] {顶点指定 Vertex Specification};
    \node (vshader) [optbox, below=of vspec, solid] {顶点着色 Vertex Shader};
    \node (tess) [optbox, below=of vshader] {曲面细分 Tessellation};
    \node (gshader) [optbox, below=of tess] {几何着色 Geometry Shader};
    \node (vpost) [box, below=of gshader] {顶点后处理 Vertex Post-Processing};
    \node (passem) [box, below=of vpost] {图元装配 Primitive Assembly};
    \node (raster) [box, below=of passem] {光栅化 Rasterization};
    \node (fshader) [optbox, below=of raster] {片段着色 Fragment Shader};
    \node (psample) [box, below=of fshader] {逐样本操作 Per-Sample Operations};
    \draw [arrow] (vspec) -- (vshader);
    \draw [arrow] (vshader) -- (tess);
    \draw [arrow] (tess) -- (gshader);
    \draw [arrow] (gshader) -- (vpost);
    \draw [arrow] (vpost) -- (passem);
    \draw [arrow] (passem) -- (raster);
    \draw [arrow] (raster) -- (fshader);
    \draw [arrow] (fshader) -- (psample);
\end{tikzpicture}
\caption{OpenGL渲染管线示意图}
\end{figure}
上述流程图中仅有标蓝色框的部分是可编程的,其余步骤均在硬件驱动中实现,仅能通过一些选项调整参数.大部分情况下需要手动编写的仅有\tbf{顶点着色器}.
\end{document}