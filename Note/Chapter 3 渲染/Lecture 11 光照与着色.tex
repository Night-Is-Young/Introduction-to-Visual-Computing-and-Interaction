\documentclass{ctexart}
\usepackage{Note}
\begin{document}
\section{光照与着色}
\subsection{材质与着色}
我们已经学习了如何在计算机中储存和表示三维物体.然而,现实中的物体除几何属性外,还存在\tbf{材质(Material)}属性.材质决定了物体表面对光的反射和吸收特性,从而影响我们最终看到的物体的颜色,纹理等视觉信息.在计算机图形学中,为几何体的表面加上材质的过程称为\tbf{着色(Shading)}.
\begin{definition}[材质与着色]\\
    \tbf{材质(Material)}是物体表面的光学属性,决定物体如何与光相互作用,也即决定物体呈现的视觉效果.\\
    \tbf{着色(Shading)}是绘制几何体表面的颜色使得它的视觉效果符合设定材质的过程.
\end{definition}
\subsection{光照}
\subsubsection{光源}
按照光源发出光线的不同,光源可以分为以下几类.
\begin{enumerate}
    \item \tbf{点光源(Point Light Source)}:点光源在空间中可以看作一个无大小的点,从该点向各个方向均匀发射光线.点光源的光强度$I$只与距离有关,并且正比于$1/r^2$,而与方向无关.
    \item \tbf{平行光源(Directional Light Source)}:平行光源发出的光线在空间中是平行的.平行光源的光强度$I$与距离无关,但与接受平面与光线方向的夹角$\theta$有关,即$I_{\text{eff}}=I\cos\theta$.
    \item \tbf{环境光(Ambient Light Source)}:环境光源表示来自环境中各个方向的漫反射光线的总和,可以认为在空间中均匀分布,因此其光强度$I$与距离和方向均无关.
\end{enumerate}
\subsubsection{反射}
与研究光照类似,我们先考虑简单情况下的反射规律.
\paragraph{漫反射}
当物体表面粗糙不平时,入射到表面的光线会被反射到各个方向,称为\tbf{漫反射(Diffuse Reflection)}.
\begin{definition}[漫反射与朗博体]\\
    \tbf{漫反射(Diffuse Reflection)}是指入射光线被物体表面反射到各个方向的现象.\\
    \tbf{朗博体(Lambertian Surface)}是表面只有完全随机的漫反射的物体.
\end{definition}
对于朗博体的漫反射而言,我们有如下简单的公式以计算其反射的光强:
\[L_d=k_aI_a+k_d\left(I_d+\dfrac{I_p}{r^2}\right)\max(\cos\theta,0)\]
其中$L_d$表示漫反射后的光强(下标$d$意为diffuse,表示漫反射), $k_a$和$k_d$分别是物体对环境光的反射率和物体表面的漫反射率.简单而言\footnote{物体表面的颜色也会影响对各种波长的光的反射率,不过也可以逐颜色分量地应用上述公式,因此通常使用的$k$包含RGB三个分量.此时,如果光源的RGB分量表示为向量$\vec{I}$,反射率表示为向量$\vec{k}$,那么最终的反射光颜色可以表示为\[\vec{I}\circ\vec{k}\]其中$\circ$表示向量逐分量相乘.因此,上述公式在各个颜色分量不同时可以写成\[\vec{L}_d=\vec{k}_a\circ\vec{I}_a+\vec{k}_d\circ\left(\vec{I}_d+\dfrac{\vec{I}_p}{r^2}\right)\max(\cos\theta,0)\]在下面的镜面反射的公式中也类似.}, $k_a$和$k_d$越小,物体表面吸收的光越多, 反射光强$L_d$也越小,物体就越暗;反之物体就越亮. $I_a,I_d$和$I_p$分别表示环境光,平行光和点光源强度, $r$表示点光源到物体的距离.\\
\indent 上述式子中的角度$\theta$表示物体指向光源的单位向量$\vec{l}$和物体表面的单位法向量$\vec{n}$的夹角,因此式子中的$\cos\theta$可以替换为$\vec{l}\cdot\vec{n}$.注意,当$\theta>\pi/2$时,光线入射到物体的背面,此时物体并不接受光照,因此我们取$\max(\cos\theta,0)$以舍弃此种情形.
\paragraph{镜面反射}
当物体表面光滑时,入射到表面的光线会按照一定的规律被反射,称为\tbf{镜面反射(Specular Reflection)}.
\begin{definition}[镜面反射]
    \tbf{镜面反射(Specular Reflection)}是指入射光线按照反射定律被反射的现象.
\end{definition}
一般而言,由于物体表面并不能完全光滑,镜面反射并不总是完全符合反射定律,而是分布于镜面反射光线$\vec{r}$附近.如果物体表面指向观察者的方向为$\vec{v}$,那么应当有$\vec{v}$与$\vec{r}$的夹角越小,反射光强越大.这样,我们可以写出镜面反射的光强计算公式:
\[L_s=k_s\left(I_d+\dfrac{I_p}{r^2}\right)\max(\vec{v}\cdot\vec{r},0)^\alpha\]
其中$L_s$表示镜面反射后的光强(下标$s$意为specular,表示镜面反射), $k_s$是物体表面的镜面反射率. $\alpha$是镜面反射指数, $\alpha$越大,视线远离镜面反射方向时光强衰减地越快,高光区域越集中,反之则越分散.理想镜面反射光线$\vec{r}$可由
\[\vec{r}=2\left(\vec{l}\cdot\vec{n}\right)\vec{n}-\vec{l}\]
给出.\\
\indent 除去这种办法之外,人们提出了一种巧妙的计算方法:定义半程向量
\[\vec{h}=\dfrac{\vec{l}+\vec{v}}{\left|\left|\vec{l}+\vec{v}\right|\right|}\]
表示光源方向$\vec{l}$与观察方向$\vec{v}$的角平分线方向.观察方向$\vec{v}$越接近$\vec{r}$,半程向量$\vec{h}$就越接近法线$\vec{n}$,因此可以将上式改写为
\[L_s=k_s\left(I_d+\dfrac{I_p}{r^2}\right)\max(\vec{n}\cdot\vec{h},0)^\beta\]
其中$\beta$也是镜面反射指数.
\paragraph{Phong光照模型和Blinn-Phong光照模型}
把前述的漫反射和镜面反射结合起来,我们就得到了常用的光照模型.
\begin{definition}[Phong光照模型]
    \tbf{Phong光照模型(Phong Reflection Model)}结合了漫反射和镜面反射,其光强计算公式为:
    \[L=k_aI_a+k_d\left(I_d+\dfrac{I_p}{r^2}\right)\max(\vec{l}\cdot\vec{n},0)+k_s\left(I_d+\dfrac{I_p}{r^2}\right)\max(\vec{v}\cdot\vec{r},0)^\alpha\]
\end{definition}
\begin{definition}[Blinn-Phong光照模型]
    \tbf{Blinn-Phong光照模型(Blinn-Phong Reflection Model)}是Phong光照模型的变种,其光强计算公式为:
    \[L=k_aI_a+k_d\left(I_d+\dfrac{I_p}{r^2}\right)\max(\vec{l}\cdot\vec{n},0)+k_s\left(I_d+\dfrac{I_p}{r^2}\right)\max(\vec{n}\cdot\vec{h},0)^\beta\]
\end{definition}
可以看出,两者的区别仅在于镜面反射部分. Blinn-Phong模型使用了半程向量$\vec{h}$,这在高光表现的效果上更平滑自然,因而更多地被使用.
\subsection{渲染的光栅化}
与二维图形一样,我们可以通过\tbf{光栅化}将虚拟的几何与材质转化为像素.这主要包含两个问题:确定像素对应的三维位置(也即正确处理遮挡关系),以及确定该像素的颜色.
\subsubsection{深度缓存}
\paragraph{画家算法}
为了确定物体的前后关系,最初想到的办法是给每个形状/物体一个额外的深度属性$z$,然后对所有形状的深度排序,按照$z$从大到小(即从远到近)的顺序进行绘制.这和画家的绘画方法类似.
\begin{definition}[画家算法]
    上述过程称作\tbf{画家算法(Painter's Algorithm)}.
\end{definition}
然而,真实世界中物体各处的深度值可能并不相同,因此可能出现复杂的遮挡关系,使得画家算法难以实现.
\paragraph{深度缓存}
为了解决这个问题,我们不妨转换思路,由记录每个物体的深度转为记录每个像素的深度值,所有像素的深度值构成\tbf{深度缓存(Depth Buffer)}.每个图形各处的深度值也不是固定的.在绘制时,我们对每个像素独立检测,如果发现等待绘制的图形上的深度小于当前像素的深度,则覆盖当前像素的颜色和深度值,否则表示图形在此处被遮挡,不更新像素.
\begin{definition}
    上述过程称作\tbf{深度缓存(Depth Buffer)}算法.
\end{definition}
深度缓存技术可以处理更复杂的情况,并且只需要维护所有像素的缓存,是一种更高效的办法.\\
\indent 然而,深度缓存技术对于半透明物体的处理并不理想,因为半透明物体叠加后的颜色仍密切取决于此处各物体的前后关系.因此,通常在这些地方仍然需要使用画家算法.
\subsubsection{着色模型}
我们现在考虑第二个问题:确定像素的颜色.这可以用各种着色模型来实现.下面给出几种常见的着色模型.
\paragraph{平面着色}
最简单而直接的办法就是使用网格的每个面的法向量进行光照计算,并为整个面赋予同样的颜色.
\begin{definition}[平面着色]
    上述过程称作\tbf{平面着色(Flat Shading)}.
\end{definition}
在平面着色中,每个面只有一个法向,因而只有一个颜色,于是我们能在渲染结果里看到很多独立而不连续的面片,但是我们还是可以直观看到整个形状上明暗的过渡以及高
光.如果要得到比较平滑的结果,就需要更精细的网格.为了解决这一问题,人们提出了许多其它的着色模型.
\paragraph{Gouraud着色}
与平面着色不同,我们可以根据每个顶点的法向量进行光照计算,并为顶点赋予颜色.然后在渲染时,通过重心插值等插值方法得出三角形面片内部各点的颜色.
\begin{definition}[Gouraud着色]
    上述过程称作\tbf{Gouraud着色(Gouraud Shading)}.
\end{definition}
Gouraud着色在曲面的大部分区域里都可以得到光滑的颜色过渡,但是在高光区域我们能明显看到三角形插值的痕迹.这是因为在高光区域附近,颜色变化随着法向量变化比较明显,并且这不是线性关系(例如,如果高光中心在三角形面片的中心,而顶点处颜色较暗,显然不能通过插值表现出中心更亮的颜色).
\paragraph{Phong着色}\footnote{注意与前面的Blinn-Phong反射模型加以区分.两者分别是反射模型和着色模型,并没有直接联系.}为了克服Gouraud着色在高光区域的不足,我们可以直接在三角形面片的\tbf{每个像素}处计算法向量,并根据该法向量计算颜色.
\begin{definition}[Phong着色]
    上述过程称作\tbf{Phong着色(Phong Shading)}.
\end{definition}
\subsection{风格化渲染}
应用于不同场景的渲染并不一定追求真实感,有时希望突出/隐去一些细节或展现特定的艺术风格.这类渲染称为\tbf{风格化渲染(Stylized Rendering)}或\tbf{非真实感渲染(Non-photorealistic Rendering, NPR)}.\\
\indent 简单而言,风格化渲染中比较重要的两个特征分别是\tbf{线}和\tbf{风格化的着色模型}.我们现在分别介绍这两种特征相关的处理方法.
\subsubsection{轮廓线提取}
物体的轮廓线对描述物体的信息非常重要.在风格化渲染中通常会对轮廓线进行特殊处理.
\begin{theorem}[轮廓线的提取方法I]
    物体的边缘位置的发现$\vec{n}$应当与视线方向$\vec{v}$近似地垂直,因此我们可以判断物体上所有满足$\left|\vec{n}\cdot\vec{v}\right|<\ep$的点,这些点就可以视作物体的轮廓线.
\end{theorem}
上述办法依赖于$\ep$的设定,并且得到的边缘并不一定粗细均匀.由此,我们还有另一种办法.
\begin{theorem}[轮廓线的提取办法II]
    我们绘制两遍几何体.第一遍只绘制背面,并且绘制的颜色为轮廓线的颜色,同时将几何体向外扩展一些;第二遍再在背面上绘制正面的几何体,如此,没有被遮挡的部分就是轮廓线.\\
    这种办法又被称作\tbf{程序化几何法}.
\end{theorem}
使用上述办法时,需要注意向外扩展的实现办法,也就是将顶点沿它的法向移动一定距离,并且这段距离在屏幕上最终呈现的长度是固定的,因此这需要考虑投影变换带来的影响;此外,绘制正面的几何体时,需要使用深度缓存,否则可能丢失部分轮廓线.
\subsubsection{风格化着色模型示例——Gooch着色}
除去用光影表现物体的形状,我们还可以使用颜色的冷暖变化来表现物体的形状.一种常见的风格化着色模型是\tbf{Gooch着色(Gooch Shading)}.
\begin{definition}[Gooch着色]
    取定冷色$k_{c}$和暖色$k_w$,物体上各点的颜色通过插值得到:
    \[k=\left(\dfrac{1+\vec{l}\cdot\vec{n}}{2}\right)k_c+\left(\dfrac{1-\vec{l}\cdot\vec{n}}{2}\right)k_w\]
    其中$\vec{l}$是物体上的点指向光源的方向, $\vec{n}$是该点表面法线的方向.
\end{definition}
\end{document}