\documentclass{ctexart}
\usepackage{Note}
\begin{document}
\section{几何变换}
\subsection{齐次坐标系}
我们在上一节中简单介绍了几何变换对应的数学操作.例如,平移可以视作向量的加法,旋转可以视作旋转矩阵与向量的乘法.但是,这些操作并不能统一表示,因而带来了额外的麻烦.例如,平移无法表示为矩阵与向量的乘法.为了解决这个问题,我们可以引入齐次坐标系的概念.\\
\indent 齐次坐标系的基本思想是,通过引入额外的维度,将所有的几何变换都表示为矩阵与向量的乘法.
\begin{definition}[齐次坐标系]
    在$n$维空间$R^n$中,点$\vec{v}=\left(\li x,n\right)$的齐次坐标表示为$\left(\li x,n,1\right)$,即在原有坐标的基础上增加一个分量恒为$1$的维度.\\
    相应地,齐次坐标系中的向量$\vec{v}=\left(\li x,n\right)$表示为$\left(\li x,n,0\right)$,即在原有坐标的基础上增加一个分量恒为$0$的维度.
\end{definition}
如此,我们可以将所有的几何变换都表示为$(n+1)\times(n+1)$的矩阵与齐次坐标向量的乘法.例如,在二维空间中的一些几何操作对应的矩阵如下:
\begin{table}[H]\centering
    \begin{tabular}{cc}
        平移&旋转\\
        $\begin{bmatrix}
            x'\\y'\\1
        \end{bmatrix}=\begin{bmatrix}
            1&0&t_x\\0&1&t_y\\0&0&1
        \end{bmatrix}\begin{bmatrix}
            x\\y\\1
        \end{bmatrix}$&$\begin{bmatrix}
            x'\\y'\\1
        \end{bmatrix}=\begin{bmatrix}
            \cos\theta&-\sin\theta&0\\\sin\theta&\cos\theta&0\\0&0&1
        \end{bmatrix}\begin{bmatrix}
            x\\y\\1
        \end{bmatrix}$
    \end{tabular}
\end{table}
\end{document}