\documentclass{ctexart}
\usepackage{Note}
\begin{document}
\section{曲线}
\subsection{曲线数学基础}
\subsubsection{曲线的表示}
一般而言,我们可以把曲线(以及以后的曲面)的表示方式分为显式表示和隐式表示.
\begin{definition}[显式表示]
    显式表示是可以直接通过表达式得到点的表示方式.例如,平面上的圆的参数方程即显示表示:
    \[\begin{cases}
        x=r\cos t\\
        y=r\sin t
    \end{cases}t\in[0,2\pi)\]
    一般的二次曲线也是显示表示:
    \[y=ax^2+bx+c\]
\end{definition}
\begin{definition}[隐式表示]
    隐式表示是指通过隐式方程来表示曲线(或曲面),而不直接给出参数到坐标的映射的表示方法.例如,平面上的圆可以表示为隐式方程
    \[f(x,y)=x^2+y^2-r^2=0\]
\end{definition}
隐式表示可以更容易地分辨曲线的内外侧,但相应地不容易直接得到曲线的形状.因此,在曲线绘制时更常用显式表示.
\subsubsection{曲线插值与基函数}
\begin{definition}[基函数]
    给定$n+1$个点$(x_0,y_0),\cdots,(x_n,y_n)$,如果存在$n+1$个函数$\phi_0(x),\cdots,\phi_n(x)$,使得
    \[\phi_i(x_j)=\begin{cases}
        1&i=j\\
        0&i\neq j
    \end{cases}\]
    那么称$\phi_i(i=0,\cdots,n)$为这些点的\tbf{基函数(Basis Function)}.对这些点插值的结果可以写作
    \[y=\sum_{i=0}^ny_i\phi_i(x)\]
    这恰好是过各点的曲线,具体形状由基函数的性质决定.
\end{definition}
\subsubsection{曲线的连续性}
\begin{definition}[参数连续性]
    设曲线的$n$阶导数在连接的两段曲线的交点处相等,则称这两段曲线在该点处是$C^n$连续的.
\end{definition}
\begin{definition}[几何连续性]
    $n$阶几何连续的定义如下:
    \begin{enumerate}[label=\arabic*.,topsep=0pt,parsep=0pt,itemsep=0pt,partopsep=0pt]
        \item $G^0$连续:两段曲线在连接点处相交.
        \item $G^1$连续:两段曲线在连接点处相交,且切线方向相同,即曲率方向相同而大小不同.
        \item $G^2$连续:两段曲线在连接点处相交,且曲率方向和大小均相同.
        \item $G^3$连续:两段曲线在连接点处相交,且曲率方向,大小和变化率均相同.
    \end{enumerate}
\end{definition}
\subsection{Bezier曲线}
\subsubsection{Bezier曲线的定义}
Bezier曲线是计算机图形学中常用的一种参数曲线,由法国工程师Pierre Bézier在20世纪60年代为汽车车身设计而开发.它们广泛应用于计算机图形学、动画、字体设计等领域.\\
\indent Bezier曲线通过一组控制点来定义.我们先来看如何构造二阶Bezier曲线.给定三个控制点$\vec{P}_0,\vec{P}_1,\vec{P}_2$(这里的粗体表示原点到这一点的向量,下同),我们可以通过两轮线性插值得到曲线.首先,由参数$t$对$\overrightarrow{P_0P_1}$和$\overrightarrow{P_1P_2}$做线性插值:
\[\vec{Q}_0(t)=(1-t)\vec{P}_0+t\vec{P}_1\]
\[\vec{Q}_1(t)=(1-t)\vec{P}_1+t\vec{P}_2\]
然后用同一个参数$t$对$\overrightarrow{Q_0Q_1}$做线性插值:
\[\vec{S}(t)=(1-t)\vec{Q}_0(t)+t\vec{Q}_1(t)\]
当$t$取遍$[0,1]$时,$\vec{S}$对应的点的集合就是二阶Bezier曲线.上述过程称作\tbf{德卡斯特里奥算法(De Casteljau's Algorithm)},效率高,编程方便,因而被广泛使用.\\
\indent 类似地, $n$阶Bezier曲线由$n+1$个控制点$\vec{P}_0,\vec{P}_1,\cdots,\vec{P}_n$通过$n$轮线性插值得到.
\begin{lstlisting}
def bezier(points: list[Vec2], t: float) -> Vec2:
    n = len(points)
    P = points.copy()
    for r in range(1, n):
        for i in range(n - r):
            P[i] = (1 - t) * P[i] + t * P[i + 1]
    return P[0]
\end{lstlisting}
\indent 下面推导Bezier曲线的显式表达式.首先考虑$n=2$的情形,将前述表达式展开可得
\[\begin{aligned}
    \vec{S}
    &= (1-t)\vec{Q}_0+t\vec{Q}_1\\
    &= (1-t)\left[(1-t)\vec{P}_0+t\vec{P}_1\right]+t\left[(1-t)\vec{P}_1+t\vec{P}_2\right]\\
    &= (1-t)^2\vec{P}_0+2t(1-t)\vec{P}_1+t^2\vec{P}_2
\end{aligned}\]
可以发现,$\vec{P}_0$,$\vec{P}_1$和$\vec{P}_2$的系数分别是$((1-t)+t)^2$进行二项式展开的系数.因此,我们可以猜测\footnote{事实上可以将这一过程与杨辉三角的构造相联系而证明.}$n$阶Bezier曲线的显式表达式为
\[\vec{S}(t)=\sum_{k=0}^{n}\begin{pmatrix}n\\k\end{pmatrix}(1-t)^kt^{n-k}\vec{P}_k\]
其中
\[B_{n,k}(t)=\begin{pmatrix}n\\k\end{pmatrix}(1-t)^kt^{n-k}\]
被称作\tbf{$n$阶伯恩斯坦多项式(Bernstein Polynomial)}.\\
\indent 我们也可以用矩阵的形式来表示Bezier曲线.例如,三阶Bezier曲线可以通过以下的形式表示:
\[\vec{S}(t)=\boldsymbol{\tau}^{\text{T}}\mat{M}_{B}\vec{u}\]
其中
\[\boldsymbol{\tau}^{\text{T}}=\begin{bmatrix}1&t&t^2&t^3\end{bmatrix}\ \ \ \ \ \mat{M}_B=\begin{bmatrix}
    1&0&0&0\\
    -3&3&0&0\\
    3&-6&3&0\\
    -1&3&-3&1
\end{bmatrix}\ \ \ \ \ \vec{u}^{\text{T}}=\begin{bmatrix}\vec{P}_0&\vec{P}_1&\vec{P}_2&\vec{P}_3\end{bmatrix}\]
\subsubsection{Bezier曲线的性质}
首先有
\[\vec{S}(0)=\vec{P}_0\ \ \ \ \ \vec{S}(1)=\vec{P}_n\]
因此Bezier曲线经过第一个和最后一个控制点.\\
\indent 其次有
\[\begin{aligned}
    \dfrac{\di\vec{S}(t)}{\di t}
    &= \dfrac{\di}{\di t}\left(\sum_{k=0}^{n}\dfrac{n!}{k!(n-k)!}(1-t)^kt^{n-k}\vec{P}_k\right) \\
    &= \sum_{k=0}^{n}\dfrac{n!}{k!(n-k)!}\left((n-k)(1-t)^kt^{n-k-1}-k(1-t)^{k-1}t^{n-k}\right)\vec{P}_k \\
    &= n\sum_{k=0}^{n}\left(B_{n-1,k}(t)-B_{n-1,k-1}(t)\right)\vec{P}_k\\
    &= n\sum_{k=0}^{n-1}B_{n-1,k}(t)(\vec{P}_{k+1}-\vec{P}_k
)
\end{aligned}\]
于是
\[\vec{S}'(0)=n\left(\vec{P}_1-\vec{P}_0\right)\ \ \ \ \ \vec{S}'(1)=n\left(\vec{P}_n-\vec{P}_{n-1}\right)\]
可见, Bezier曲线在端点处的切线方向分别沿着$\overrightarrow{P_0P_1}$和$\overrightarrow{P_{n-1}P_n}$的方向.
\begin{theorem}[Bezier曲线与端点的关系]
    设Bezier曲线的控制点依次为$P_0,\cdots,P_n$,则它经过$P_0$和$P_n$,且在端点处的切线方向分别沿着$\overrightarrow{P_0P_1}$和$\overrightarrow{P_{n-1}P_n}$的方向.
\end{theorem}
因此,如果希望两条Bezier曲线首尾相接且切线连续,只需让它们的端点重合且相邻的控制点共线即可.\\
\indent 于是,三阶Bezier曲线是比较常用的,既可以自由控制曲线的形状,复杂度也不高.\\
\indent Bezier曲线的另一个重要性质是\tbf{凸包性质}.
\begin{definition}[凸包性质]
    设$\vec{P}_0,\vec{P}_1,\cdots,\vec{P}_n$为Bezier曲线的控制点,那么Bezier曲线完全包含在由这些控制点构成的凸多边形内.
\end{definition}
\begin{proof}
    由于$B_{n,k}(t)\geqslant 0$且$\displaystyle\sum_{k=0}^nB_{n,k}(t)=1$,因此$\vec{S}(t)$是控制点的凸组合,从而在凸包内.
\end{proof}
\subsubsection{Bezier曲面}
同样地,我们可以用类似的方法绘制Bezier曲面.这需要用到两个参数$u,v$.例如,三阶Bezier曲面可以表示为
\[\vec{S}(u,v)=\sum_{i=0}^{3}\sum_{j=0}^{3}B_{3,i}(u)B_{3,j}(v)\vec{P}_{ij}=\vec{u}^{\text{T}}\mat{M}_B\vec{P}\mat{M}_B^{\text{T}}\vec{v}\]
其中
\[\mat{P}=\begin{bmatrix}
    \vec{P}_{00}&\cdots&\vec{P}_{03}\\
    \vdots&\ddots&\vdots\\
    \vec{P}_{30}&\cdots&\vec{P}_{33}
\end{bmatrix}\ \ \ \ \ \vec{v}=\begin{bmatrix}
    1\\v\\v^2\\v^3
\end{bmatrix}\]
\subsection{样条曲线}
\begin{definition}[样条曲线]
    \tbf{样条曲线(Spline Curve)}是计算机图形学和数值分析中常用的一类分段定义的多项式函数,用于平滑地插值或逼近一组离散数据点.
\end{definition}
给定点列$\{\vec{P}_0,\cdots,\vec{P}_m\}$,其中$\vec{P}_i=\left(x_i,y_i\right)$.样条曲线的确定实际上就是在每个区间$[x_i,x_{i+1}]$上构造一个多项式$S_i(x)$,使其经过各点并保持一定的连续性.常见的样条曲线有以下几种.
\subsubsection{三次样条曲线}
三次样条曲线是比较简单而常用的.假定在每一段$\left\{\vec{P}_i,\vec{P}_{i+1}\right\}$上定义参数$t\in[0,1]$,并且假定曲线上的点可以表示为
\[\vec{S}_i(t)=\vec{a}_it^3+\vec{b}_it^2+\vec{c}_it+\vec{d}_i\]
曲线一共有$4m$个未知数,我们需要$4m$个方程来确定这些未知数.\\
\indent 首先,曲线需要经过各个点,因此有
\[\vec{S}_i(0)=\vec{P}_i,\ \vec{S}_i(1)=\vec{P}_{i+1}\]
其次,为了保持光滑性,我们需要保证相邻段的切线方向一致,因此有
\[\vec{S}'_i(1)=\vec{S}'_{i+1}(0)\]
最后,为了保持曲线的平滑性,我们还需要保证相邻段的二阶导数一致,因此有
\[\vec{S}''_i(1)=\vec{S}''_{i+1}(0)\]
上述三个条件一共有$4m-2$个方程.为此,我们还需要自行指定两个边界条件,例如令曲线在端点处的二阶导数为$0$,称作\tbf{自然边界条件}:
\[\vec{S}''_0(0)=\mbf{0},\ \vec{S}''_{m-1}(1)=\mbf{0}\]
求解这一线性方程组即可得到三次样条曲线的参数.\\
\indent 从上述过程可以看出,三次样条曲线是插值曲线,并且具有$C^2$连续性.然而,改变任意一个控制点会影响整条曲线的形状,因此三次样条曲线不具有局部性.
\begin{definition}[局部性]
    如果改变一个控制点只会影响曲线的局部形状而不影响整体形状,则称该插值曲线具有\tbf{局部性}.
\end{definition}
\subsubsection{Hermite样条曲线}
为了避免控制点对远端曲线形状的影响,我们可以对三次样条曲线的构造做改进.我们不要求各段曲线的二阶导连续,而是指定各顶点$\vec{P}_i$处的切线方向为$\vec{p}_i$,于是对于$\left(\vec{P}_i,\vec{P}_{i+1}\right)$之间的三次函数有
\[\vec{S}_i(0)=\vec{P}_i\ \ \ \ \ \vec{S}_i(1)=\vec{P}_{i+1}\ \ \ \ \ \vec{S}'_i(0)=\vec{p}_i\ \ \ \ \ \vec{S}'_i(1)=\vec{p}_{i+1}\]
这可以直接解得
\[\vec{S}_i(t)=\left(2t^3-3t^2+1\right)\vec{P}_i+\left(t^3-2t^2+t\right)\vec{p}_i+\left(-2t^3+3t^2\right)\vec{P}_{i+1}+\left(t^3-t^2\right)\vec{p}_{i+1}\]
这就是三次Hermite样条曲线.其它阶数的Hermite样条曲线也可以类似地构造.\\
\indent Hermite样条曲线各控制点的导数方向可以自行指定,也可以由控制点得到.著名的Catmull-Rom样条曲线就是一种特殊的Hermite样条曲线,其切线方向由相邻控制点决定,即
\[\vec{p}_i=\dfrac{\vec{P}_{i+1}-\vec{P}_{i-1}}{2}\]
相比三次样条曲线, Hermite样条曲线同样是插值曲线,但同时具有局部性.相应地,它在光滑程度上有所牺牲,只有$C^1$连续性.此外,它并不需要解线性方程组,结果已经一定,计算效率更高\footnote{Microsoft PowerPoint提供的曲线工具就是三阶Hermite样条曲线,用户可以自由编辑每个控制点的位置和对应的切线.}.
\subsubsection{B样条曲线}
与前面两种样条曲线相比,B样条曲线不是插值曲线,与Hermite样条曲线一样具有局部性,但光滑性更好.\\
\indent 选定参数区间$[a,b]$内一递增的数列$a=t_0<t_1<\cdots<t_m=b$,则$n$次B样条曲线可以表达为
\[\vec{b}(t)=\sum_{i=0}^{m}N_{i,n}(t)\vec{P}_i\]
其中$N_{i,n}(t)$为B样条基函数,通过递归定义:
\[N_{i,0}(t)=\left\{\begin{array}{l}
    1,\ t_i\leqslant t<t_{i+1}\\
    0,\ \text{otherwise}
\end{array}\right.\]
\[N_{i,p}(t)=\dfrac{t-t_i}{t_{i+p}-t_i}N_{i,p-1}(t)+\dfrac{t_{i+p+1}-t}{t_{i+p+1}-t_{i+1}}N_{i+1,p-1}(t)\]
不难看出,$N_{i,p}(t)$是$p$阶的分段多项式,并且在连接处保持了$C^{p-1}$连续性,因此$n$阶B样条曲线具有$C^{n-1}$连续性.此外,$N_{i,p}(t)$在$\left[t_i,t_{i+p+1}\right)$非$0$,因此改变$\vec{P}_i$仅会影响$\left[t_i,t_{i+p+1}\right)$区间内的曲线形状. $p$越大,各个基函数和控制点影响的范围就越大,局部性就越差.\\
\indent B样条曲线的一大优势是其灵活性.我们可以分段选择不同阶数的基函数,也可以选取重复节点$t_i=t_{i+1}=\cdots=t_{k}$使得曲线其余部分连续性不变的情况下构造出尖锐的转折.
\subsection{曲线光栅化}
\subsubsection{中点圆算法}
中点圆算法(Midpoint Circle Algorithm)是Bresenham教授在1962年提出的一种高效的光栅化圆的算法,其主要思路与Bresenham直线算法类似.\\
\indent 为了方便考虑,我们假定要绘制的圆以原点为中心,半径为$r$.由于圆的对称性,我们只需计算第一象限的$1/8$圆,然后将结果对称变换到其他象限即可.\\
\indent 考虑$\dfrac\pi4$到$\dfrac\pi2$的$1/8$圆,我们从$(0,r)$开始按顺时针方向绘制.设当前绘制的点为$P_i(x_i,y_i)$,那么下一个点$P_{i+1}$也仅有两种可能: $P_{i+1}=(x_i+1,y_i)$或$P_{i+1}=(x_i+1,y_i-1)$.我们用中点$P_i'=\left(x_i+1,y_i-\dfrac12\right)$来判断下一个点的位置,如果$P_i'$在圆内,就向右移,否则向右下移动.圆的隐式方程为
\[F(x,y)=x^2+y^2-r^2=0\]
同样,如果$f\left(P_{i}'\right)<0$,那么$P_i'$在圆内,否则在圆外.我们可以只对$F\left(P_i'\right)$进行更新,显然每次判断点的移动情况和像素的移动情况一样,因此
\[F\left(P_{i+1}'\right)=F\left(x_{P_i'}+\Delta x,y_{P_i'}+\Delta y\right)=F\left(P_i'\right)+2x_{P_i'}\Delta x+\left(\Delta x\right)^2+2y_{P_i'}\Delta y+\left(\Delta y\right)^2\]
其中$(\Delta x,\Delta y)=(1,0)$或$(1,-1)$.由于$P_{i}'$的坐标带有分数,因此我们将上述式子转写为关于$P_{i}$的坐标的式子,即
\[F\left(P_{i+1}'\right)=F\left(P_i'\right)+2x_i\Delta x+2\Delta x+\left(\Delta x\right)^2+2y_i\Delta y-\Delta y+\left(\Delta y\right)^2\]
于是向右更新像素(即$(\Delta x,\Delta y)=(1,0)$时)需要将判断函数加上$2x_i+3$,向右下更新像素(即$(\Delta x,\Delta y)=(1,-1)$时)需要将判断函数加上$2x_i-2y_i+5$.初始条件下有
\[F\left(1,r-\dfrac12\right)=\dfrac54-r\]
既然我们涉及的计算都是整数计算,因此将初始值设为$1-r$并不改变判断正负的结果.当然,如果半径$r$是浮点数,就需要先计算上式后取整了.\\
\indent 综上,我们可以将中点圆算法写成下面的程序:
\begin{lstlisting}[language=Python]
def draw_circle(xc: int, yc: int, r: int):
    x, y = 0, r
    F = 1 - r
    while x <= y:
        draw(xc + x, yc + y); draw(xc + y, yc + x)
        draw(xc - x, yc + y); draw(xc - y, yc + x)
        draw(xc + x, yc - y); draw(xc + y, yc - x)
        draw(xc - x, yc - y); draw(xc - y, yc - x)
        if F < 0: # F(P_i') < 0
            F += 2 * x + 3 # F(P_{i+1}) = F(P_i) + 2 * x_i + 3
        else: # F(P_i') >= 0
            y -= 1
            F += 2 * (x - y) + 5 # F(P_{i+1}) = F(P_i) + 2 * (x_i - y_i) + 5
        x += 1
\end{lstlisting}
\end{document}