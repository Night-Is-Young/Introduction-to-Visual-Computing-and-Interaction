\documentclass{ctexart}
\usepackage{Note}
\begin{document}
\section{绘图}
\subsection{光栅化和直线绘制}
\subsubsection{光栅化}
\begin{definition}[光栅化]
    将图形转换为像素的过程称为\tbf{光栅化(rasterization)}.
\end{definition}
光栅化的思想在三维绘制(即渲染)中也会用到.
\subsubsection{直线绘制}
假定我们有一块分辨率为$W\times H$的屏幕,并且希望在屏幕上绘制直线$y=kx+b(x_0\leqslant x_1)$,起止点为$\left(x_0,y_0\right),\left(x_1,y_1\right)$.为了简化问题,我们假设$0\leqslant k\leqslant 1$.
\paragraph{基于浮点运算的直线绘制法}
最简单的想法是遍历所有$x\in\left[x_0,x_1\right]$,计算对应的$y$值,然后将$(x,y)$处的像素点填上对应的颜色.
\begin{lstlisting}[language=Python]
def draw_line(x0:int, y0:int, k:float, b:float, x1:int):
    for x in range(x0, x1 + 1):
        y = k * x + b
        draw(x,round(y))
\end{lstlisting}
\indent 然而,浮点数的运算开销是较大的.为了避免浮点数的乘法,又由于我们的直线是等间隔采样的,因此可以使用累加法.
\begin{lstlisting}[language=Python]
def draw_line(x0:int, y0:int, x1:int, y1:int):
    k = (y1 - y0) / (x1 - x0)
    y = y0
    for x in range(x0, x1 + 1):
        draw(x, round(y))
        y += k
\end{lstlisting}
这就是直线绘制的\tbf{DDA算法(Digital Differential Analyzer)}.\\
\indent 但是,上面的算法中仍然包含了浮点数的加法.\paragraph{Bresenham直线算法} Bresenham教授在1962年提出了著名的\tbf{布雷森汉姆直线算法(Bresenham's Line Algorithm)},能够在只计算整数加减法的情况下获得和DDA算法相同的结果.\\
\indent 令直线的起终点为$P_0,P_1$.令$\overrightarrow{P_0P_1}$顺时针旋转$90^\circ$形成的向量$\vec{n}=\left(y_1-y_0,x_0-x_1\right)$,即直线的法向量,那么直线上任意一点$P$就满足
\[F(P)=\vec{n}\cdot\overrightarrow{P_0P}=0\]
这就是直线的隐式方程.给出任意一点$Q$,判断$Q$在直线的上方还是下方,只需判断$F(Q)$的符号即可.不难得出,如果$F(Q)$为负,那么$Q$在直线上方,反之则在直线下方.\\
\indent 假定已经绘制了直线上的一点$\left(x_i,y_i\right)$.对于斜率$k\in(0,1)$的直线,下一个像素的坐标仅有两种可能: $\left(
x_i+1,y_i\right)$或$\left(x_i+1,y_i+1\right)$. Bresenham认为如果上述两点的中点$\left(x_i+1,y_i+\dfrac12\right)$在直线上方,就绘制$\left(x_i+1,y_i\right)$,否则绘制$\left(x_i+1,y_i+1\right)$.这种判断方法显然比较符合直觉.\\
\indent 现在我们考虑具体的实现方法.对于已经绘制的点$P_i\left(x_i,y_i\right)$,需要用点$P_i'\left(x_i+1,y_i+\dfrac12\right)$判断下一个像素.为了避免直接计算$F\left(P_i'\right)$(否则我们又要计算乘法),我们考虑下面的递推方法:
\[F\left(P_i'\right)=\vec{n}\cdot\overrightarrow{P_{i-1}'P_i'}=\vec{n}\cdot\left(\overrightarrow{P_0P_{i-1}'}+\overrightarrow{P_{i-1}'P_i'}\right)=F\left(P_{i-1}'\right)+\vec{n}\cdot\boldsymbol{\delta}\]
其中$\boldsymbol{\delta}=(1,0)$或$(1,1)$(判断点的移动和绘制点的移动显然是同步的).这样,画完之后我们可以每次更新$F\left(P_i'\right)$的值以进行下一个像素位置的判断.$\vec{n}\cdot\boldsymbol{\delta}$可以在循环前就算好,不必重复.递推的初值为
\[F\left(P_0'\right)=F\left(P_0\right)+\vec{n}\cdot\left(1,\dfrac12\right)=\left(y_1-y_0\right)-\dfrac12\left(x_1-x_0\right)\]
为了避免$1/2$带来的浮点运算,我们可以把$F$放大$2$倍.\\
\indent 总结而言,我们可以将上面的方法写成下面的程序:
\begin{lstlisting}[language=Python]
def draw_line(x0: int, y0: int, x1: int, y1: int):
    y = y0
    dx, dy = 2 * (x1 - x0), 2 * (y1 - y0)
    dydx, F = dy - dx, dy - dx // 2
    for x in range(x0, x1 + 1):
        draw(x, y)
        if F < 0: # F(P_i') < 0
            F += dy # F(P_{i+1}') = F(P_i') + n * delta_0
        else: # F(P_i') >= 0
            y += 1
            F += dydx # F(P_{i+1}') = F(P_i') + n * delta_1
\end{lstlisting}
\paragraph{所有直线的绘制}
我们在前面只讲了斜率$k\in[0,1]$的直线的绘制.对于其他斜率的直线,通过适当的变换也可以做到一样的效果.\\
\indent 首先根据起终点计算直线的斜率$k$,如果$|k|\leqslant 1$,那么遍历$x$坐标并按照上述算法更新$y$坐标(如果$k<0$,更新$y$时需将其递减);如果$|k|>1$,那么交换前述$x$和$y$的角色即可.代码实现如下:
\begin{lstlisting}[language=Python]
def draw_line(x0: int, y0: int, x1: int, y1: int):
    f = (y1 - y0) < (x1 - x0)
    if not f: x0, y0, x1, y1 = y0, x0, y1, x1
    if x0 > x1: x0, y0, x1, y1 = x1, y1, x0, y0
    y, sy = y0, 1 if (y1 > y0) else -1
    dx, dy = 2 * (x1 - x0), 2 * (y1 - y0)
    dydx, F = dy - dx, dy - dx // 2
    for x in range(x0, x1 + 1):
        if f: draw(x, y)
        else: draw(y, x)
        if F < 0:
            F += dy
        else: 
            y += sy
            F += dydx
\end{lstlisting}
\subsection{多边形填充}
\subsubsection{多边形的光栅化}
三角形是最简单的多边形,也是多边形的基本组成部分.最简单的办法是枚举所有可能的像素点,判断其是否在三角形内.
\begin{lstlisting}[language=Python]
def draw_triangle(x0: int, y0: int, x1: int, y1: int, x2: int, y2: int):
    xmin, xmax = min(x0, x1, x2), max(x0, x1, x2)
    ymin, ymax = min(y0, y1, y2), max(y0, y1, y2)
    for x in range(xmin, xmax + 1):
        for y in range(ymin, ymax + 1):
            if inside(x, y, x0, y0, x1, y1, x2, y2):
            draw(x, y)
\end{lstlisting}
要判断点$Q$是否在三角形内,可以将边按逆时针顺序排列,并将边逆时针旋转$90^\circ$得到的法向量记作$\vec{N}_i(i=0,1,2)$.各$\vec{N}_i$都指向三角形内部,于是只要各$F_i(Q)=\vec{N}_i\cdot\overrightarrow{P_iQ}>0$即可说明$Q$在三角形内.\\
\indent 然而,上面的方法对于每个点都要计算三次点积,效率不高.注意到我们每次总是在按行填充像素,因此只要维护每行的起点$x_L$和终点$x_R$即可.我们也不必通过点积计算点的位置,只要根据边的斜率和上一行的起终点更新即可(即使用DDA算法更新起终点).这就是\tbf{扫描线算法(Scanline Algorithm)}.\\
\indent 对于多边形,也可以用扫描线算法绘制,但需要注意非凸的情形.此时我们需要根据交点的数目按奇偶规则填充像素.假定交点为$x_1,\cdots,x_n$,那么需要填充的部分为$x_{2k-1}$到$x_{2k}$的部分.这可以通过简单的画图证明.当然,也可以将多边形分解成多个三角形进行分别绘制.\\
\indent 扫描线算法相比于简单的三角形算法更高效,然而在现代计算机上,其实使用的最多的是并行版的简单算法.由于三角形在图形学中的重要性,现代GPU搭载了用于三角形内外检测的专门模块,从而在硬件上实现并行的简单算法.
\subsubsection{多边形的插值}
\paragraph{颜色插值}
在绘制多边形时,我们有时希望多边形的颜色是渐变的.在绘制线段时,假定两个顶点$\left(x_0,y_0\right)$和$\left(x_1,y_1\right)$的颜色为$c_0$和$c_1$,我们可以通过线性插值得到线段上$(x,y)$的颜色:
\[c=\dfrac{(x_1-x)c_0+(x-x_0)c_1}{x_1-x_0}=c_0(1-t)+c_1t,t=\dfrac{x-x_0}{x_1-x_0}\]
对多边形进行插值时,我们可以应用扫描线算法,每一行内的颜色由两个端点插值得到,而端点的颜色又可以由对应的边的端点插值得到.上述插值方法即\tbf{双线性插值(Bilinear Interpolation)}.\\
\indent 在不使用扫描线算法的情况下,我们需要单独确定每个点的颜色.我们可以用\tbf{重心坐标(Barycentric Coordinate)}来做插值.设三角形的顶点为$A,B,C$,那么对于三角形内的任意一点$P$,令
\[\alpha=\dfrac{S_{\triangle PBC}}{S_{\triangle PBC}}\ \ \ \ \ \beta=\dfrac{S_{\triangle PAC}}{S_{\triangle PBC}}\ \ \ \ \ \gamma=\dfrac{S_{\triangle PAB}}{S_{\triangle PBC}}\]
根据中学所学的平面向量知识不难得出
\[\alpha+\beta+\gamma=1\ \ \ \ \ \alpha\overrightarrow{PA}+\beta\overrightarrow{PB}+\beta\overrightarrow{PC}=\mbf{0}\]
于是我们可以把$P$处的颜色表示为
\[c=\alpha c_A+\beta c_B+\gamma c_C\]
有关上面两种插值方式有以下结论.
\begin{theorem}[重心插值和双线性插值的等价性]
    重心插值是线性的,即在任意一处发生相同位移时颜色的改变值一样.特别地,三角形的重心插值和双线性插值等价.
\end{theorem}
\begin{proof}
    仍然设三角形的顶点为$A\left(x_0,y_0\right),B\left(x_1,y_1\right),C\left(x_2,y_2\right)$,并设$P\left(x,y\right)$在三角形内.重心坐标的表示要求
    \[\left\{\begin{array}{l}
        \alpha_P(x_0-x)+\beta_P(x_1-x)+\gamma_P(x_2-x)=0\\
        \alpha_P(y_0-y)+\beta_P(y_1-y)+\gamma_P(y_2-y)=0\\
        \alpha_P+\beta_P+\gamma_P=1
    \end{array}\right.\]
    这一线性方程组具有唯一解
    \[\left\{\begin{array}{l}
        \alpha_P=\dfrac{\left(y-y_1\right)\left(x_2-x_1\right)-\left(x-x_1\right)\left(y_2-y_1\right)}{\left(y_0-y_1\right)\left(x_2-x_1\right)-\left(x_0-x_1\right)\left(y_2-y_1\right)}\\
        \beta_P=\dfrac{\left(y-y_2\right)\left(x_0-x_2\right)-\left(x-x_2\right)\left(y_0-y_2\right)}{\left(y_1-y_2\right)\left(x_0-x_2\right)-\left(x_1-x_2\right)\left(y_0-y_2\right)}\\
        \gamma_P=\dfrac{\left(y-y_0\right)\left(x_1-x_0\right)-\left(x-x_0\right)\left(y_1-y_0\right)}{\left(y_2-y_0\right)\left(x_1-x_0\right)-\left(x_2-x_0\right)\left(y_1-y_0\right)}
    \end{array}\right.\]
    $\alpha_P,\beta_P$和$\gamma_P$都是关于$x,y$的线性函数,因此$P$处的颜色$c_P$可以写为如下形式:
    \[c_P=k_1x+k_2y+C\]
    其中$k_1,k_2,C$与$x,y$无关.因此,重心插值是线性的.\\
    \indent 现在来证明重心插值和双线性插值等价.如果将颜色$c$作为三维空间中的$z$坐标,那么这两种插值方式都在空间中描绘了一个平面.它们都过三角形的顶点对应的$\left(x_i,y_i,c_i\right)$,因而对应同一平面,从而等价.
\end{proof}
\paragraph{多边形的拉伸}
我们以图片的拉伸作为多边形插值的例子.考虑一张四边形图片,顶点为$A,B,C,D$,现将这张图片拉伸到屏幕上的四边形$A'B'C'D'$.我们可以在原图建立直角坐标系,每个顶点对应$\left(u_i,v_i\right)$,在拉伸后顶点和边上的坐标值不变(这可以先由边的线性插值得到),然后对四边形内的点进行双线性插值.上面的方法称作\tbf{UV坐标映射(UV Mapping)}.\\
\indent 尽管三角形的重心插值和双线性插值等价,但四边形没有唯一的插值方法.直接对其双线性插值和分解成两个三角形进行插值得到的结果不同.
\end{document}