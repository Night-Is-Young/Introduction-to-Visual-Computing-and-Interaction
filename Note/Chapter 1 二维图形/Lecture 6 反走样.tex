\documentclass{ctexart}
\usepackage{Note}
\begin{document}
\section{反走样}
\subsection{反走样的理论基础}
根据Fourier变换的性质,可以得出:对连续函数$f_0(x)$以频率$f_s$进行采样,最终得到的离散函数$f_1(x)$的频谱$F_1(f)$是把原函数的频谱$F_0(f)$以$f_s$为周期进行周期延拓的结果.\\
\indent 记$f_0(x)$的截止频率为$f_0$(即其频域$F_0(f)$在$f>f_0$的区域上均为$0$),如果$f_s>2f_0$,那么将$F_0(f)$按照$f_s$为周期平移时就不会出现重叠,也就可以准确地复现$f_0(x)$.否则,当$F_1(f)$中出现重叠时,就出现了错误的频率信号,\tbf{走样(Aliasing)}就发生了.
\begin{theorem}[Nyquist-Shannon采样定理]
    对于一个截止频率为$f$的连续信号,如果以大于$2f$的采样频率对其进行采样,则可以准确重建该信号.否则,会发生混叠现象,无法准确重建信号.
\end{theorem}
因此,一切走样几乎都是因为采样频率$f_s$不足导致的.除去简单地增高采样频率外,我们需要考虑如何在有限的采样频率下减轻或避免走样的发生.
\subsubsection{低通滤波}
将采样前的信号去除高频成分(即大于$f_s/2$的部分)即可避免走样的发生.这步骤就是\tbf{低通滤波(Low-pass Filtering)}.
\begin{definition}[低通滤波]
    低通滤波是指通过滤波器去除信号中的高频成分,只保留低频成分的过程.
\end{definition}
\subsubsection{MIPmap}

\end{document}