\documentclass{ctexart}
\usepackage{Note}
\begin{document}
\section{显示}
\subsection{二维显示技术}
\subsubsection{矢量与像素}
\begin{definition}[矢量显示]
    \tbf{矢量显示}是指直接控制电子束在荧光屏上按照一定路径绘制图形的技术.
\end{definition}
矢量显示器可以直接绘制线条和曲线,因此在显示几何图形和文字时具有较高的精度和清晰度.然而,矢量显示器通常无法显示复杂的图像,因为这时的线条数目太多,难以处理.
\begin{definition}[像素显示]
    \tbf{像素显示}是指将显示区域划分为小的单元格(即\tbf{像素(Pixel)}),通过控制每个像素的亮度和颜色来形成图像的技术.
\end{definition}
像素显示一般对应于阴极射线显示器的按行扫描的形式(即\tbf{光栅显示技术}).不管图形复杂与否,显示的开销是不变的,这更有利于应对复杂的图形.
\begin{definition}[分辨率]
    屏幕显示的像素数目称为\tbf{分辨率(Resolution)}.
\end{definition}
常见的分辨率有$1080\text{p}(1920\times1080)$, $2\text{K}(2048\times1080)$, $4\text{K}(3840\times2160)$等\footnote{有关$2\text{K}$和$4\text{K}$的命名规则比较复杂,可以参考}.\\
\indent 像素显示亦有缺点.用离散的像素点近似连续的图形,就会导致锯齿效应.我们在后面的章节中会介绍反走样(即抗锯齿)技术.
\subsubsection{显示原理}
\begin{definition}[帧与帧率]
    \tbf{帧(Frame)}是指屏幕上显示的单张图像,而\tbf{帧率(Frame Rate)}是指每秒钟显示的帧数,通常以FPS(Frames Per Second)为单位.
\end{definition}
\begin{definition}[刷新率]
    \tbf{刷新率(Refresh Rate)}是指屏幕每秒钟更新的次数,通常以Hz为单位.
\end{definition}
提高帧率可以使画面更流畅.像素显示技术可以通过隔行扫描的方式来提高帧率.
\begin{definition}[隔行扫描]
    \tbf{隔行扫描}是指每次只更新屏幕的一半像素,即先更新所有奇数行,然后更新所有偶数行的显示技术.
\end{definition}
采用隔行扫描的方式可以提高显示设备的刷新率以及画面的帧率.
\subsubsection{颜色显示}
对于现代大多数显示设备,颜色显示的原理是一致的:通过红,绿,蓝三种颜色的色光混合出RGB颜色空间中的各种颜色.对于各种类型的屏幕,其实现方式不同.
\begin{other}[分类]{现代显示器分类及显示颜色的原理}
    \begin{enumerate}[label=\arabic*.,topsep=0pt,parsep=0pt,itemsep=0pt,partopsep=0pt]
        \item CRT(阴极射线显示):通过电子束轰击荧光屏上的红,绿,蓝三种荧光粉来显示颜色.
        \item LCD(液晶显示):使用电压控制偏振光的偏转角度,从而控制白光透过偏振片的强弱,在经过红绿蓝三种颜色的滤光片之后混合出目标颜色.
        \item LED(发光二极管)/OLED(有机发光二极管):通过控制红,绿,蓝三种二极管的亮度来混合出各种颜色.OLED使用有机发光材料制作发光二极管,相较LED有更高的对比度和更鲜明的颜色.
    \end{enumerate}
\end{other}
\begin{definition}[色域]
    显示设备能够显示的颜色范围称作\tbf{色域(Gamut)}.
\end{definition}
不同显示技术的显示设备有不同的色域,例如OLED通常具有更广的色域,能够显示更丰富的颜色.
\subsubsection{亮度显示}
人眼能感知的亮度范围很广,而传统摄像和显示设备的亮度显示的动态范围有限,显示效果有时不佳.这就需要HDR技术.
\begin{definition}[高动态范围显示]
    \tbf{高动态范围显示(HDR,High Dynamic Range)}是指能够显示更宽亮度范围的图像显示技术.
\end{definition}
通过复杂的算法扩展和压缩图像各区域的亮度, HDR允许在亮部和暗部表现更多的细节,从而接近人眼感知的真实画面.\\
\indent 此外,早期CRT显示设备的亮度和电压信号成非线性关系,但摄像机等设备产生的信号是正比于光强的,因此亮度显示可能有失真.CRT显示设备的亮度$I$与电压$V$的关系通常为
\[I\propto V^\gamma\]
其中$\gamma$通常在$2.2$左右,这会导致暗部更偏暗.为解决这个问题,我们需要对图像各像素的亮度值做校正,即
\[l\to l^{1/\gamma}\]
这样会把图片先变亮,然后再传入显示设备,就能减小失真影响.
\begin{definition}[伽马校正]
    上述对图像各像素的亮度值做校正的过程称为\tbf{伽马校正(Gamma Correction)}.
\end{definition}
尽管现在常用的显示设备,如LCD显示器和OLED显示器等,都不再具有上述特性,但伽马校正已经被标准化于数字图像处理步骤中而被一直沿用. sRGB颜色空间直接规定$\gamma=2.2$,各种显示器需要做相应的伽马校正.\\
\indent 在渲染时,由于我们模拟的是现实世界的光照,因此使用线性颜色空间是最合适的,只需在输出图像时做伽马校正即可.渲染中所用的各贴图也应保存在线性颜色空间中,无需做伽马校正.
\subsection{三维显示技术}
\subsubsection{三维显示的基本原理}
人类获得立体感的途径很多.
\begin{other}[原理]{三维显示的基本原理}
    \begin{enumerate}[label=\arabic*.,topsep=0pt,parsep=0pt,itemsep=0pt,partopsep=0pt]
        \item 双眼视差:由于人眼之间有一定距离,从而使得双眼看到的图像略有不同.大脑通过处理这两幅图像的差异来感知深度信息.
        \item 运动视差:当我们移动头部或身体时,近处的物体相对于远处的物体移动得更快.大脑利用这种运动引起的视差来推断物体的距离.
        \item 透视:平行线在远处会汇聚于一点(即\tbf{消失点}),物体随着距离增加而变小.大脑利用这些线索来理解空间关系.
        \item 光影和遮挡:光照和阴影提供了关于物体形状和位置的重要信息.此外,遮挡关系也帮助我们理解哪些物体在前面,哪些在后面.
        \item 焦点调节:人眼通过调节晶状体的形状来聚焦不同距离的物体.大脑利用这种调节信息来感知深度.
    \end{enumerate}
\end{other}
\subsubsection{三维显示技术}
根据前面的原理,常见的三维显示技术主要有以下几种.
\begin{other}[分类]{常见三维显示技术}
    \begin{enumerate}[label=\arabic*.,topsep=0pt,parsep=0pt,itemsep=0pt,partopsep=0pt]
        \item 立体显示:通过为每只眼睛提供略有不同的图像来模拟双眼视差.
        \item 全息显示:利用干涉和衍射原理记录和重现光波的完整信息,从而生成三维图像.
        \item 光场显示:通过捕捉和再现光线在空间中的传播方向和强度来创建三维图像.
        \item 虚拟现实(VR):通过头戴式显示器和运动追踪技术创建沉浸式三维环境.
        \item 增强现实(AR):将虚拟对象叠加到现实世界中显示.
    \end{enumerate}
\end{other}
需要注意的是,立体显示技术与三维显示技术不同,它只能通过双眼视差模拟固定视点的立体感,而三维显示技术一般能提供各个视点的试图,以求真实地还原三维空间.立体显示技术的主要目标就是让左右眼看到不同的图像.
\begin{other}[分类]{常见立体显示技术}
    \begin{enumerate}[label=\arabic*.,topsep=0pt,parsep=0pt,itemsep=0pt,partopsep=0pt]
        \item 偏振光立体显示:使用偏振光技术,在同一屏幕上显示具有两种偏振方向地图像,通过偏振眼镜分别给每只眼睛传递对应图像.
        \item 微柱透镜立体显示:在屏幕前放置微柱透镜阵列,通过透镜对光的折射使得每只眼睛只能看到特定的像素,从而实现立体效果.
        \item 光屏障立体显示:在屏幕前放置细小光屏障,通过屏障的遮挡使得每只眼睛只能看到特定的像素,从而实现立体效果.
    \end{enumerate}
\end{other}
后两种技术都是裸眼3D技术,不需要佩戴眼镜,但视点位置有限制,过大的视点移动会导致立体效果丧失,甚至造成不适.\\
\indent 如果根据追踪到的视线方向将光屏障立体显示中的光屏障进行动态改变,就能实现动态裸眼3D技术.这就是三维显示技术的一种.
\end{document}