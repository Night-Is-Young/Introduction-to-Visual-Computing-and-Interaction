\documentclass{ctexart}
\usepackage{Note}
\title{\textbf{Lab 5 报告}}
\author{蒋锦豪 2400011785}
\date{\today}
\begin{document}
\maketitle
\section*{Task 1: Parallel Coordinates Visualization}
结构体\code{CoordinateStates}主要包含以下部分:\\
\indent 首先对读入的数据\code{data}进行预处理和存储,将数据保存在二维数组中,并且记录各字段的最小值\code{minData}和最大值\code{maxData}.这部分的代码如下:
\begin{lstlisting}[language=c++]
std::vector<std::string>        labelName = { "cylinders", "displacement", "horsepower", "weight", "acceleration", "mileage", "year" };
std::vector<int>                labelIdx  = { 0, 1, 2, 3, 4, 5, 6 };
int                             mainIdx   = 0; // main axis index
int                             dataNum   = 7, dataSize;
std::vector<std::vector<float>> dataTable; 
std::vector<float>              minData, maxData;
CoordinateStates(std::vector<Car> const& data) {
    dataSize = data.size();
    for (auto car : data) {
        std::vector<float> carData;
        carData.push_back(car.cylinders);
        ...
        dataTable.push_back(carData);
    }
    minData = dataTable[0];
    maxData = dataTable[0];
    for (size_t i = 0; i < dataSize; ++i) {
        for (size_t j = 0; j < dataNum; ++j) {
            minData[j] = std::min(dataTable[i][j], minData[j]);
            maxData[j] = std::max(dataTable[i][j], maxData[j]);
        }
    }
}
\end{lstlisting}
然后通过线性插值的方法形成各数据点在绘图区域的坐标,这部分的代码如下:
\begin{lstlisting}[language=c++]
std::vector<std::vector<std::pair<float, float>>> dataPlotPos;
const float minY = 0.15f, maxY = 0.85f;
void ConstructPlotPos() {
    dataPlotPos.clear();
    for (size_t i = 0; i < dataSize; ++i) {
        std::vector<std::pair<float, float>> line;
        for (int j = 0; j < dataNum; ++j) {
            int   k    = labelIdx[j];
            float yPos = minY + (maxData[k] - dataTable[i][k]) / (maxData[k] - minData[k]) * (maxY - minY);
            line.push_back(std::pair<float, float>((2 * j + 1) / (2.0f * dataNum), yPos));
        }
        dataPlotPos.push_back(line);
    }
}
\end{lstlisting}
然后是绘图函数\code{Paint}的实现.根据生成的坐标列表\code{dataPlotPos}按照主轴\code{mainIdx}的大小进行排序,然后用预先设置的两种颜色\code{red}和\code{blue}进行插值得到数据对应的折线的颜色;随后给各坐标轴进行矩形框高亮,对主轴使用红色,对其余轴使用灰色;最后在各坐标轴上下标出数据和文本作为注释.这部分的代码如下:
\begin{lstlisting}[language=c++]
const float yShift = 0.05f, rectWidth = 0.015f, axisWidth = 2.0f, frameWidth = 1.5f;
const float textShift = 0.025f, lineHeight = 0.01f;
void Paint(Common::ImageRGB & input) {
    ConstructPlotPos();
    int idx = mainIdx;
    std::sort(dataPlotPos.begin(), dataPlotPos.end(), [idx](const std::vector<std::pair<float, float>> & a, const std::vector<std::pair<float, float>> & b) {
        return a[idx].second < b[idx].second;
    });

    SetBackGround(input, backwhite);
    for (size_t i = 0; i < dataSize; ++i) {
        float     s        = 1.0f * i / dataSize;
        float     t        = 0.5 + 4 * std::pow(s - 0.5, 3);
        glm::vec4 curColor = t * blue + (1 - t) * red;
        glm::vec2 from(dataPlotPos[i][0].first, dataPlotPos[i][0].second), to;
        for (size_t j = 1; j < dataNum; ++j) {
            to = glm::vec2(dataPlotPos[i][j].first, dataPlotPos[i][j].second);
            DrawLine(input, curColor, from, to, 0.1f);
            from = to;
        }
    }
    for (size_t i = 0; i < dataNum; ++i) {
        float     centerX = (i * 2 + 1) * 1.0f / (dataNum * 2);
        glm::vec2 leftTop(centerX - rectWidth / 2, minY - yShift), size(rectWidth, maxY - minY + 2 * yShift);
        DrawLine(input, darkgray, glm::vec2(centerX, minY - yShift), glm::vec2(centerX, maxY + yShift), 2.0f);
        DrawFilledRect(input, (i == idx) ? backred : lightgray, leftTop, size);
        DrawRect(input, frontwhite, leftTop, size, frameWidth);
    }
    for (size_t i = 0; i < dataNum; ++i) {
        float centerX = (i * 2 + 1) * 1.0f / (dataNum * 2);
        PrintText(input, black, glm::vec2(centerX, minY - yShift - textShift * 2), lineHeight, labelName[labelIdx[i]]);
        PrintText(input, black, glm::vec2(centerX, minY - yShift - textShift), lineHeight, std::to_string((int) maxData[labelIdx[i]]));
        PrintText(input, black, glm::vec2(centerX, maxY + yShift + textShift), lineHeight, std::to_string((int) minData[labelIdx[i]]));
    }
}
\end{lstlisting}
最后是对鼠标交互的处理.基本的设计思路是左键单击轴可以改变主轴,右键依次单击两个不同的轴可以将两个轴进行交换.在右键单击第一个轴时,用\code{labelSwap}记录是否正在执行交换操作.同时,在交换完成后,需要判断主轴是否被交换,若是则需要同时修改主轴的位置.具体的代码实现如下:
\begin{lstlisting}[language=c++]
const float clickShift = 0.06f;
bool        labelSwap  = false;
glm::vec2   startPos, endPos;
int         startIdx = -1, endIdx = -1;
void ResetLabelSwap() {
    startIdx  = -1;
    endIdx    = -1;
    labelSwap = false;
}

bool Update(InteractProxy const & proxy) {
    if (! proxy.IsHovering()) return false;
    if (proxy.IsClicking()) {
        glm::vec2 clickPos = proxy.MousePos();
        for (size_t i = 0; i < dataNum; ++i) {
            float centerX = (i * 2 + 1) * 1.0f / (dataNum * 2);
            if (std::abs(clickPos.x - centerX) < clickShift && clickPos.y > (minY - clickShift) && clickPos.y < (maxY + clickShift)) {
                mainIdx      = i;
                return true;
            }
        }
    }
    if (proxy.IsClicking(false)) {
        if (!labelSwap) {
            startPos = proxy.MousePos();
            for (size_t i = 0; i < dataNum; ++i) {
                float centerX = (i * 2 + 1) * 1.0f / (dataNum * 2);
                if (std::abs(startPos.x - centerX) < clickShift && startPos.y > (minY - clickShift) && startPos.y < (maxY + clickShift)) {
                    startIdx = i;
                }
            }
            if (startIdx != -1) labelSwap = true;
            return false;
        } else {
            endPos  = proxy.MousePos();
            for (size_t i = 0; i < dataNum; ++i) {
                float centerX = (i * 2 + 1) * 1.0f / (dataNum * 2);
                if (std::abs(endPos.x - centerX) < clickShift && endPos.y > (minY - clickShift) && endPos.y < (maxY + clickShift)) {
                    endIdx = i;
                }
            }
            if (endIdx != -1 && startIdx != endIdx) {
                if (startIdx == mainIdx) mainIdx = endIdx;
                if (endIdx == mainIdx) mainIdx = startIdx;
                std::swap(labelIdx[startIdx], labelIdx[endIdx]);
                ResetLabelSwap();
                return true;
            } else {
                ResetLabelSwap();
                return false;
            }
        }
    }
}
\end{lstlisting}
上述几个部分就是结构体\code{CoordinateStates}的内容.在\code{PaintParallelCoordinates}函数中实现所有功能的代码如下:
\begin{lstlisting}[language=c++]
bool PaintParallelCoordinates(Common::ImageRGB & input, InteractProxy const & proxy, std::vector<Car> const & data, bool force) {
    // your code here
    // for example: 
    //   static CoordinateStates states(data);
    //   SetBackGround(input, glm::vec4(1));
    //   ...
    static CoordinateStates states(data);                  // initialize
    bool                    change = states.Update(proxy); // update according to user input
    if (! force && ! change) return false;                 // determine to skip repainting
    states.Paint(input);                                   // visualize
    return true;
}
\end{lstlisting}
\end{document}